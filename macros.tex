\usepackage{graphicx}
%\usepackage[demo]{graphicx}

\usepackage{wrapfig}

\setlength{\textfloatsep}{8pt plus 0.0pt minus 0.0pt}
\setlength{\floatsep}{0pt plus 0.0pt minus 0.0pt}
\setlength{\intextsep}{4pt plus 0.0pt minus 0.0pt}
\setlength{\abovecaptionskip}{0pt}

\usepackage{multirow}

\usepackage{scrextend}
\usepackage{sistyle}
\SIthousandsep{,}

\usepackage{amsthm,amssymb,amsmath}
\usepackage[mathscr]{eucal}


\usepackage{pgfgantt}
\usepackage{enumitem}
\setlist[enumerate]{label={\bf \alph*)},itemsep=0pt,topsep=5pt,parsep=0pt,leftmargin=.60in}
\setlist[itemize]{label=\bsf{--},itemsep=0pt,topsep=0pt,parsep=0pt,leftmargin=.20in}
%\usepackage[singlelinecheck=false,font={small,sf}]{subfig} 
\usepackage[font={sf}]{caption}
\usepackage[font={small,sf}]{subcaption}
%\usepackage[singlelinecheck=false,font={small}]{subfig} 
%\usepackage[font={normalsize}]{caption}

%\newcommand{\sam}[1]{{\normalsize{{({Sam:\ }\color{blue}#1})}}}

\usepackage{soul}
\setul{3.5pt}{.4pt}

\usepackage{bibentry} % \nobibliography, 

\usepackage{comment}

\usepackage[%hyperref
    pdftex,
    hypertexnames,%
    citecolor=darkblue,%
    colorlinks=true,%
    linkcolor=darkblue,%
    urlcolor=darkblue%
]{hyperref}

\usepackage{nameref}

%\usepackage[numbers,sort&compress]{natbib}
%\setlength{\bibsep}{0.0pt}
%\bibpunct{[}{]}{,}{n}{}{;}
\usepackage[                   % use biblatex for bibliography
  backend=bibtex8,             %   - use bibtex8 backend 
  bibencoding=utf8,            %   - use auto file encode
%  style=numeric,            %   - use alphabetic (or numeric) bib style
%  style=numeric-comp,            %   - use numeric bib style & [1,2,3] -> [1-3]
  style=alphabetic,            %   - use alphabetic (or numeric) bib style
%  style=authoryear,            %   - use alphabetic (or numeric) bib style
%  bibstyle=authoryear,            %   - use alphabetic (or numeric) bib style
%  citestyle=authoryearbrack,
%  style=draft,                %   - use alphabetic (or numeric) bib style
  natbib=true,                 %   - allow natbib commands
  hyperref=true,               %   - activate hyperref support
%  backref=true,                %   - activate backrefs
  isbn=false,                  %   - don't show isbn tags
%  url=false,                   %   - don't show url tags
  url=true,                    %   - don't show url tags
  doi=true,                    %   - show doi tags
  urldate=long,                %   - display type for dates
  firstinits=true,
  terseinits=false,
%  sorting=none,
  sortcites=true,
  maxnames=3,%
  minnames=1,%
  maxbibnames=25,%
  minbibnames=15,%
  maxcitenames=2,%
  mincitenames=1%
]{biblatex}
\usepackage[utf8]{inputenc} %% Handles special characters in the bibliography
\usepackage[T1]{fontenc}
\setcounter{biburlnumpenalty}{100}

%\ExecuteBibliographyOptions{maxbibnames=6,minbibnames=6,maxcitenames=1,mincitenames=1}

\DeclareCiteCommand{\fullcite}
  {\usebibmacro{prenote}}
  {\usedriver
     {\defcounter{minnames}{6}%
      \defcounter{maxnames}{6}}
     {\thefield{entrytype}}}
  {\multicitedelim}
  {\usebibmacro{postnote}}

%\setlength{\bibitemsep}{10pt}

\renewbibmacro*{doi+eprint+url}{%
  \iftoggle{bbx:doi}
    {\printfield{doi}}
    {}%
  \newunit\newblock
  \iftoggle{bbx:eprint}
    {\iffieldundef{doi}{\iffieldundef{url}{\printfield{eprint}}{}}{}}
    {}%
  \newunit\newblock
  \iftoggle{bbx:url}
    {\iffieldundef{doi}{\printfield{url}}{}}
    {}}


% Double spacing, if you want it.
% \def\dsp{\def\baselinestretch{2.0}\large\normalsize}
% \dsp

% If the Grad. Division insists that the first paragraph of a section
% be indented (like the others), then include this line:
% \usepackage{indentfirst}

\usepackage{xcolor}
\definecolor{darkblue}{HTML}{0000CC}
\definecolor{darkgreen}{HTML}{00A800}
\definecolor{darkpurple}{HTML}{6300b4}
\definecolor{darkred}{HTML}{8B0000}
\definecolor{darkgray}{HTML}{666666}
\definecolor{_mage}{HTML}{912830}
\definecolor{_cyan}{HTML}{31837a}
\definecolor{_purp}{HTML}{49425c}

\usepackage{fullpage}


%\newcommand{\bsf}[1]{{\sf\textbf{#1}}}
\newcommand{\bsf}[1]{{\textbf{#1}}}
\newcommand{\paragraphsf}[1]{{\sf\textbf{#1}}}

\usepackage[compact]{titlesec}
%\titleformat{\section}{\Large\bfseries\sffamily\centering}{\thesection}{1em}{}
%\titleformat{\subsection}{\large\bfseries\sffamily\centering}{\thesubsection}{1em}{}
%\titleformat{\subsubsection}{\bfseries\sffamily\centering}{\thesubsubsection}{1em}{}
%\titleformat{\paragraph}[runin]{\bfseries\sffamily}{\theparagraph}{1em}{}
\titleformat{\section}{\Large\bfseries\centering}{\thesection}{1em}{}
\titleformat{\subsection}{\large\bfseries\centering}{\thesubsection}{1em}{}
\titleformat{\subsubsection}{\bfseries\centering}{\thesubsubsection}{1em}{}
\titleformat{\paragraph}[runin]{\bfseries}{\theparagraph}{1em}{}

\titlespacing{\section}{0em}{0.05em}{.00em}
\titlespacing{\subsection}{0em}{.25em}{.00em}
\titlespacing{\subsubsection}{0em}{.25em}{.00em}

\usepackage{fancyhdr}
\pagestyle{fancy}
\fancyhf{}
\setlength{\headheight}{20pt}
\setlength{\footskip}{20pt}
\renewcommand{\headrulewidth}{0.4pt}
\renewcommand{\footrulewidth}{0.4pt}

%\newcommand{\citesf}[1]{{\sf \cite{#1}}}
\newcommand{\citesf}[1]{{\cite{#1}}}
\newcommand{\citesfs}[2]{{\sf \cite[#1]{#2}}}
%\newcommand{\fig}[1]{{\sf Figure~\ref{fig:#1}}}
%\newcommand{\figref}[1]{{\sf Figure~\ref{fig:#1}}}
%\newcommand{\tabref}[1]{{\sf Table~\ref{tab:#1}}}
%\newcommand{\secref}[1]{Section~\ref{sec:#1}}
\newcommand{\fig}[1]{{ Figure~\ref{fig:#1}}}
\newcommand{\figref}[1]{{ Figure~\ref{fig:#1}}}
\newcommand{\tabref}[1]{{ Table~\ref{tab:#1}}}
\newcommand{\secref}[1]{Section~\ref{sec:#1}}

%\newcommand{\sect}[1]       {\pagebreak\section{#1}}
%\newcommand{\sects}[1]      {\pagebreak\section*{#1}}
%\newcommand{\subsect}[1]    {\pagebreak\subsection{#1}}
%\newcommand{\subsects}[1]   {\pagebreak\subsection*{#1}}
%\newcommand{\subsubsect}[1] {\subsubsection{#1}}
%\newcommand{\subsubsects}[1]{\subsubsection*{#1}}
\newcommand{\sect}[1]       {\section{#1}}
\newcommand{\sects}[1]      {\section*{#1}}
\newcommand{\subsect}[1]    {\subsection{#1}}
\newcommand{\subsects}[1]   {\subsection*{#1}}
\newcommand{\subsubsect}[1] {\subsubsection{#1}}
\newcommand{\subsubsects}[1]{\subsubsection*{#1}}

\usepackage{nameref}


\newcommand{\refs}[1]{
{\small\sf
\begin{itemize}[itemsep=0pt,topsep=0pt,parsep=0pt]
#1
\end{itemize}
}
}

\newenvironment{items}
{\begin{itemize}
\setlength{\itemsep}{5pt}
\setlength{\parskip}{0pt}
\setlength{\parsep}{0pt}
}
{\end{itemize}}


\usepackage{url}
\urlstyle{sf}

\usepackage{tocloft}
\renewcommand{\cftsecfont}{\sffamily}
\renewcommand{\cftsubsecfont}{\sffamily}
\renewcommand{\cftsubsubsecfont}{\sffamily}
\renewcommand{\cftsecafterpnum}{\vskip0pt}
\setlength{\cftbeforesecskip}{0pt}
%\renewcommand{\cftsubsecafterpnum}{\vskip0pt}

\newcommand{\cf}[1]{(cf.~#1)}
\newcommand{\set}[1]{\left\{ #1 \right\}}
\newcommand{\paren}[1]{\left( #1 \right)}
\newcommand{\abs}[1]{\left| #1 \right|}
\newcommand{\norm}[1]{\left\| #1 \right\|}
\newcommand{\pw}[1]{\left\{\begin{array}{ll} #1 \end{array}\right. }
\newcommand{\mat}[2]{\left(\begin{array}{#1} #2 \end{array}\right)}
\newcommand{\vf}[1]{\frac{\partial\ }{\partial #1}}
\newcommand{\vfof}[2]{\frac{\partial #1}{\partial #2}}
\newcommand{\bd}{\partial}
\newcommand{\pd}{\partial}
\newcommand{\id}{\text{id}\,}
\newcommand{\rank}[1]{\text{rank}\,#1}
\newcommand{\td}[1]{\tilde{#1}}
\newcommand{\quot}[1]{\widetilde{#1}}
\newcommand{\eqn}[1]{\begin{equation*}\begin{aligned} #1 \end{aligned}\end{equation*}}
\newcommand{\eqnn}[1]{\begin{equation}\begin{aligned} #1 \end{aligned}\end{equation}}
\newcommand{\sm}{\setminus}
\newcommand{\into}{\rightarrow}
\newcommand{\goesto}{\rightarrow}
\newcommand{\gamgo}{\stackrel{\gamma}{\rightarrow}}
\newcommand{\inc}{\hookrightarrow}
\newcommand{\rx}[1]{#1^\eps}

\newcommand{\R}{\mathbb{R}}
\newcommand{\T}[1]{\bsf{T#1}}
\newcommand{\Tr}{\top}
\newcommand{\Hb}{\mathbb{H}}
\newcommand{\N}{\mathbb{N}}
\newcommand{\m}{\mathcal}
\newcommand{\e}{\mathscr}
\newcommand{\eps}{\epsilon}
\newcommand{\vphi}{\varphi}
\newcommand{\veps}{\varepsilon}
\newcommand{\err}{\veps}
\newcommand{\rk}{\operatorname{rank}}
\newcommand{\Int}[1]{\operatorname{Int}#1}
\newcommand{\sgn}[1]{\operatorname{sign}\left( #1 \right)}

% Banjanin Burden
\newcommand{\Tstate}{v}
\newcommand{\Tinp}{w}
\newcommand{\param}{p}
\newcommand{\state}{x}
\newcommand{\run}{r}
\newcommand{\inp}{u}
\newcommand{\optstate}{\xi}
\newcommand{\optinp}{\mu}
\newcommand{\Param}{\e{P}}
\newcommand{\State}{\e{X}}
\newcommand{\Run}{\R}
\newcommand{\Inp}{\e{U}}
\newcommand{\ssState}{X}
\newcommand{\ssInp}{U}
\newcommand{\cost}{c}
\newcommand{\controlcost}{c}
\newcommand{\designcost}{c}
\newcommand{\Ell}{\e{L}}
\newcommand{\val}{\nu}
\newcommand{\policy}{\pi}
%\newcommand{\policy}{\mu}
\newcommand{\flow}{\phi}
\newcommand{\sen}{S}
\newcommand{\senstat}{\sigma}
\newcommand{\thresh}{\veps}
\newcommand{\penal}{\mu}

\newcommand{\Parameters}{X}
\newcommand{\parameter}{x}
%\newcommand{\Behaviors}{\text{Behaviors}}
%\newcommand{\behavior}{\text{behavior}}
\newcommand{\Behaviors}{B}
\newcommand{\behavior}{b}
\newcommand{\robot}{\text{robot}}
\newcommand{\anchor}{\text{anchor}}
\newcommand{\template}{\text{model}}
\newcommand{\reduction}{\pi}
\newcommand{\performance}{\rho}

\newcommand{\D}{D}
\newcommand{\C}{C}
\newcommand{\I}{I}
\newcommand{\PC}{PC}

\newcommand{\HCPS}{HCPS}
\newcommand{\CPS}{CPS}
\newcommand{\DRC}{DRC}
\newcommand{\AI}{AI}
\newcommand{\LfD}{LfD}
\newcommand{\PbD}{PbD}
\newcommand{\SmartCities}{Smart Cities}
\newcommand{\DOF}{DOF}
\newcommand{\BMI}{BMI}
\newcommand{\UWECE}{{UW--ECE}}
\newcommand{\UWEE}{{UW--EE}}
\newcommand{\UW}{{UW}}
\newcommand{\UWCOE}{{UW--COE}}
\newcommand{\BRL}{BRL}
\newcommand{\NSF}{NSF}
\newcommand{\NASA}{NASA}
\newcommand{\TLX}{TLX}
\newcommand{\NRI}{NRI}
\newcommand{\backemf}{back--EMF}
\newcommand{\TAF}{TAF}
\newcommand{\CSNE}{CSNE}
\newcommand{\IRB}{IRB}
\newcommand{\AAV}{AAV}
\newcommand{\AUV}{AUV}
\newcommand{\VO}{VO}
\newcommand{\ERC}{ERC}
\newcommand{\REU}{REU}
\newcommand{\ROS}{ROS}
\newcommand{\JSON}{JSON}
\newcommand{\Python}{Python}
\newcommand{\SimTK}{SimTK}
\newcommand{\CRII}{CRII}
\newcommand{\TRUST}{TRUST}
\newcommand{\STEM}{STEM}
\newcommand{\URM}{URM}
\newcommand{\SAS}{SAS}
\newcommand{\KTW}{K--16}
\newcommand{\STARS}{STARS}
\newcommand{\STEP}{STEP}

\newcommand{\apriori}{\emph{a priori}}
\newcommand{\naive}{na\"{i}ve}
\newcommand{\naively}{na\"{i}vely}

\newcommand{\aim}[1]{{\sf\S\ref{sec:#1}}}
\newcommand{\aimn}[1]{Aim \S\ref{sec:#1}: \nameref{sec:#1}}

\newcommand{\Pmap}{Poincar\'{e} map}

\AtBeginDocument{%
 \abovedisplayskip=9pt plus 3pt minus 9pt
 \abovedisplayshortskip=0pt plus 3pt
 \belowdisplayskip=9pt plus 3pt minus 9pt
 \belowdisplayshortskip=7pt plus 3pt minus 4pt
}

\newtheorem{definition}{Definition}
\newtheorem{remark}{Remark}
\newtheorem{assumption}{Assumption}
\newtheorem{lemma}{Lemma}
\newtheorem{example}{Example}
\newtheorem{theorem}{Theorem}
\newtheorem{proposition}{Proposition}
\newtheorem{corollary}{Corollary}

\newcommand{\defn}[1]{\begin{definition} #1 \end{definition}}
\newcommand{\rem}[1]{\begin{remark} #1 \end{remark}}
\newcommand{\assump}[1]{\begin{assumption} #1 \end{assumption}}
\newcommand{\lem}[1]{\begin{lemma} #1 \end{lemma}}
\newcommand{\ex}[1]{\begin{example} #1 \end{example}}
\newcommand{\thm}[1]{\begin{theorem} #1 \end{theorem}}
\newcommand{\prop}[1]{\begin{proposition} #1 \end{proposition}}
\newcommand{\pf}[1]{\begin{proof} #1 \end{proof}}
\newcommand{\cor}[1]{\begin{corollary} #1 \end{corollary}}

%\urlstyle{sf}
%\bibliographystyle{unabuser}
%\nobibliography*% %load from \bibliography command, below
%\nobibliography{refs} % load from specified bib file

\usepackage{tikz}
\usetikzlibrary{arrows}
\usetikzlibrary{calc}
\usetikzlibrary{decorations.pathreplacing}
\usetikzlibrary{shapes.multipart}
\usepackage{relsize}
\tikzset{fontscale/.style={font=\relsize{#1}}}

\def\sambox#1#2#3{
  \begin{scope}[shift={#1}]
    \draw[rounded corners,style=solid,line width=3.,color=#2,fill=#2!10!white]  (-4.5,+2.2) rectangle (+5.5,+7.0);
    \node[color=#2!90!black] (a) at (0.5,+2.675) {{#3}};
  \end{scope}
}

\def\samtxt#1#2#3{
  \begin{scope}[shift={#1}]
    \node[color=#2] (a) at (0.0,+1.7) {#3};
  \end{scope}
}

\def\samarrow#1#2#3#4{
  \begin{scope}[shift={#1}]
    \path[draw,#2,line width=4.0,color=#3] (-6.0,0.0) -- (+6.0,0.0);
    \node[color=#3,shape=rectangle,draw,line width=2,fill=white] (per) at (0.0,0.0) {#4};
  \end{scope}
}

\def\samrarrow#1#2#3#4#5{
  \begin{scope}[shift={#1}]
    \path[draw,#2,line width=8.0,color=#3] (0.0,0.0) -- (#5,0.0);
    \node[color=#3,shape=rectangle,draw,line width=2,fill=white] (per) at (#5/2.2,0.0) {#4};
  \end{scope}
}

\def\samlarrow#1#2#3#4{
  \begin{scope}[shift={#1}]
    \path[draw,#2,line width=8.0,color=#3] (0.0,0.0) -- (+7.,0.0);
    \node[color=#3,shape=rectangle,draw,line width=2,fill=white] (per) at (3.5,0.0) {#4};
  \end{scope}
}

